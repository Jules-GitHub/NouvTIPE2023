\definecolor{pbblue}{rgb}{100,100,100}% color for the progress bar and the circle

\makeatletter
\def\progressbar@progressbar{} % the progress bar
\newcount\progressbar@tmpcounta% auxiliary counter
\newcount\progressbar@tmpcountb% auxiliary counter
\newdimen\progressbar@pbht %progressbar height
\newdimen\progressbar@pbwd %progressbar width
\newdimen\progressbar@rcircle % radius for the circle
\newdimen\progressbar@tmpdim % auxiliary dimension

\progressbar@pbwd=300pt
\progressbar@pbht=1pt
\progressbar@rcircle=2.5pt

% the progress bar
\def\progressbar@progressbar{%
\hspace{-200pt}
    \progressbar@tmpcounta=\insertframenumber
    \progressbar@tmpcountb=\inserttotalframenumber
    \progressbar@tmpdim=\progressbar@pbwd
    \multiply\progressbar@tmpdim by \progressbar@tmpcounta
    \divide\progressbar@tmpdim by \progressbar@tmpcountb

  \begin{tikzpicture}[remember picture,overlay]
    \draw[pbblue!100,line width=\progressbar@pbht]
      (-150pt, 6pt) -- ++ (\progressbar@pbwd,0pt);

    \filldraw[pbblue!100] %
      (\the\dimexpr\progressbar@tmpdim-\progressbar@rcircle\relax-150pt,6pt) circle (\progressbar@rcircle);
  \end{tikzpicture}%
}

\addtobeamertemplate{footline}{}
{%
  \begin{beamercolorbox}[wd=\paperwidth,ht=4ex,center,dp=1ex]{white}%
    \progressbar@progressbar%
  \end{beamercolorbox}%
}
\makeatother